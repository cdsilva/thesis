% !TEX root = thesis.tex


In science and engineering, it is becoming increasingly common to explore and analyze large, complex data sets.
%
Much of the data collected today is extremely high-dimensional, capturing detailed phenomena at the microscopic level.
%
However, many systems exhibit coherent macroscopic dynamics described by only a few effective degrees of freedom, and uncovering those relevant degrees of freedom can aid in and accelerate modeling efforts.

This dissertation addresses data collected from dynamical systems, with the goal of uncovering a parsimonious parameterization of the data which captures the governing macroscopic dynamics.
%
We use diffusion maps, a manifold learning algorithm which is flexible enough to capture nonlinear structure in a data set, to analyze such data.
%
However, traditional manifold learning algorithms, such as diffusion maps, must be modified to provide informative analysis of data from complex dynamical systems.

This thesis work adapts the diffusion maps algorithm to address four issues which are prevalent throughout dynamical systems.
%
We first analyze multiscale dynamical data, where slow dynamical modes are obscured by fast, large-magnitude noise.
%
We then present a methodology for constructing consistent embeddings, so that data collected via different observations of the same underlying dynamical system can be merged and combined using a common low-dimensional description.
%
Next, we discuss reconstructing the dynamics of a developmental process using fixed snapshots which must be both registered into a consistent frame of reference and temporally ordered.
%
Finally, we present a method for extracting the true dimensionality of a data set, an essential task for reduced modeling in complex dynamical systems, using manifold learning. 