In science and engineering, it is becoming increasingly common to deal with large, complex data sets. 
%
Much of the data collected today is extremely high-dimensional, capturing detailed phenomena at the microscopic level. 
%
However, many systems exhibit coherent macroscopic dynamics described by only a few effective degrees of freedom, and extracting these relevant degrees of freedom can aid in and accelerate modeling efforts. 

In this dissertation, we focus on data collected from dynamical systems.
%
Our goal is to uncover a parameterization of the data which captures the underlying dynamics.
%
We use diffusion maps, a manifold learning algorithm which is flexible enough to capture nonlinear structure in a data set, to analyze such data. 

The focus of this work is adapting these traditional manifold learning algorithms to analyze data from complex dynamical systems. 
%
We address four issues which are prevolent thorughout much dynamical data. 
%
We first discuss the issue of multiscale dynamical data, where slow dynamical modes are obscured by fast, large-magnitude noise. 
%
We then discuss constructing consistent embeddings, so that data collected via different observations of the same underlying dynamical system can be merged and combined using a common low-dimensional description. 
%
Next, we discuss reconstructing the dynamics of a developmental process using fixed snapshots which must first be registered into a consistent frame of reference. 
%
Finally, we present a method for extracting the true dimensionality of an underlying manifold, an essential task for reduced modeling in complex dynamical systems. 