% !TEX root = ../thesis.tex

\chapter{Conclusions \label{ch:conclusion}}

Modeling and analysis of dynamical systems is essential for a wide range of disciplines and applications. 
%
However, constructing an informative and simple model from {\em a priori} knowledge is often difficult, especially for new systems which have not yet been well-studied. 
%
In these instances, it is often advantageous to utilize data-driven techniques to construct models.
%
Often the first step is to determine the appropriate variables in which to write the models. 

The focus of this dissertation is using data-driven techniques to analyze data from dynamical systems, in order to gain insight into the appropriate variables for writing descriptive models. 
%
We use diffusion maps, a nonlinear manifold learning technique, to analyze data.
%
We adapt the diffusion maps algorithms to analyze several different types of data common in the study of dynamical systems. 
%
We discuss multiscale data, merging several data sets, imaging data, and detecting the underlying dimensionality of the data. 

For the multiscale data applications, there are both theoretical and experimental avenues to persue. 
%
As was acknowledged previously, the errors in the estimation due to finite sampling are ignored in our analysis, and should be incorporated into future work to yield a complete understanding of the methodology. 
%
Furthermore, the analysis presented was only for covariance estimation using ``bursts.''
%
It is important to also analyze the case of estimating the covariance from a signle trajectory.


Many of these ideas can be extended and utilized for a variety of applications. 
%
The diffusion maps variables can be used as the ``correct'' macroscopic variables in the equation-free framework. 
%
